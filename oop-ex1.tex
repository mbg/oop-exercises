\documentclass[10pt,a4paper]{exam}
\usepackage[latin1]{inputenc}
\usepackage{amsmath}
\usepackage{amsfonts}
\usepackage{amssymb}
\usepackage{graphicx}
\usepackage{titlesec}
\usepackage{hyperref}
\usepackage{fancyeq}
\usepackage{tikz}
%\usepackage{tikz-uml}
\usepackage{mathpartir}
\usetikzlibrary{shapes,arrows,backgrounds,positioning}
\usepackage{graphicx,xcolor}
\usepackage{geometry}
\usepackage{everysel}

\tikzset{
  treenode/.style = {align=center, inner sep=0pt, text centered,
    font=\sffamily},
  arn_n/.style = {treenode, circle, black, font=\sffamily\bfseries, draw=black,
    fill=white, text width=1.5em},% arbre rouge noir, noeud noir
  arn_r/.style = {treenode, circle, red, draw=red, 
    text width=1.5em, very thick},% arbre rouge noir, noeud rouge
  arn_x/.style = {treenode, rectangle, draw=black,
    minimum width=0.5em, minimum height=0.5em}% arbre rouge noir, nil
}

\usepackage[sc]{mathpazo}
\linespread{1.05}         % Palatino needs more leading (space between lines)
\usepackage[T1]{fontenc}

% some format settings
% for hard-bound final submission, use:
%\setlength{\oddsidemargin}{4.6mm}     % 30 mm left margin - 1 in
% for soft-bound version and techreport, use instead:

\setlength{\oddsidemargin}{-0.4mm}    % 25 mm left margin - 1 in
\setlength{\evensidemargin}{\oddsidemargin}
\setlength{\topmargin}{-5.4mm}        % 20 mm top margin - 1 in
\setlength{\textwidth}{160mm}         % 20/25 mm right margin
\setlength{\textheight}{237mm}        % 20 mm bottom margin
\setlength{\headheight}{5mm}
\setlength{\headsep}{5mm}
\setlength{\parindent}{0mm}
\setlength{\parskip}{\medskipamount}
\renewcommand\baselinestretch{1.2} % thesis format (not needed for techreport)
% don't let large figures hijack entire pages
\renewcommand\topfraction{.9}
\renewcommand\textfraction{.1}
\renewcommand\floatpagefraction{.8}

\pagestyle{headandfoot}
%\pointsinrightmargin
%\pointname{ marks}
%\marginpointname{ marks}

\marksnotpoints 

\definecolor{campurple}{HTML}{862D91} 
\definecolor{campurpledark}{HTML}{2A185C}

\hypersetup{  
  urlcolor=campurple,
  linkcolor=campurple,
  colorlinks=true  
}

\titlelabel{\llap{\thetitle\quad}}

\newcommand {\lbrac} {\makebox[0pt]{[\kern-1ex[}}
\newcommand {\rbrac} {\makebox[0pt]{]\kern-1ex]}}
\newcommand{\denote}[1]{\lbrac~#1~\rbrac}


\def\mystrut(#1,#2){\vrule height #1pt depth #2pt width 0pt} 

\titlespacing*{\section}{0pt}{0pt}{0pt}

\begin{document}

\newcommand{\course}{Object-Oriented Programming}
\newcommand{\week}{1}
\newcommand{\topics}{From Functional to Object-Oriented}

\everymath{\color{campurpledark}}
\everydisplay{\color{campurpledark}}

%\vspace{-15pt}

%\begin{center}
%\emph{Complete SECTION 1 and ONE other section.}
%\end{center}

%\begin{center}
%\emph{Answer SECTION 1 and TWO other sections.}
%\end{center}

\marksnotpoints
\pointsdroppedatright
\marksnotpoints
\marginpointname{ \points}

\begin{center}
\Large {\color{campurpledark} \course} \\[-0.2cm]
\LARGE \textbf{\color{campurpledark} Exercise \week: \topics} \\
\end{center}

{\color{campurple}\hrule}

\vspace{0.5cm}

\begin{questions}

\section*{From functional to object-oriented programming}

\question Consider the following definition of the factorial function in ML:
\begin{displaymath}
\mathbf{fun}~\mathit{factorial}~n = \mathbf{if}~n = 0~\mathbf{then}~1~\mathbf{else}~n * \mathit{factorial}~(n-1)
\end{displaymath}
\begin{parts}
\part[3] Translate this function into a corresponding Java method. \droppoints 
\part[4] Suggest \emph{two} reasons why the corresponding Java method may fail for large values of $n$ and propose a better implementation of the factorial function in Java which addresses those issues. \droppoints 
\end{parts}
\question Below is the skeleton for a Java method which should initialise a unit matrix of size $n \times n$. 
\begin{displaymath}
\mathbf{public}~\mathbf{static}~\mathbf{float}[][]~\mathit{unit}(\mathbf{int}~n)~\{~???~\}
\end{displaymath}
\begin{parts}
\part[3] Complete the definition by replacing the $???$ with the missing code. \droppoints 
\part[4] Explain potential pitfalls with the $\mathit{unit}$ method in \emph{e.g.} C++ and suggest ways to address them. \droppoints 
\end{parts}
\question[4] Below is the skeleton for a Java method which transposes a $n \times n$ matrix of values of type $\mathbf{float}$. Complete the definition by replacing the $???$ with the missing code. Your solution should use $\mathcal{O}(1)$ additional space.
\begin{displaymath}
\mathbf{public}~\mathbf{static}~\mathbf{void}~\mathit{transpose}(\mathbf{float}[][]~a)~\{~???~\}
\end{displaymath} \droppoints 

\section*{References and pointers}

\question[2] What is the calling convention used by ML and how does it work? \droppoints 
\question[2] Describe \emph{one} calling convention other than the one used by ML. \droppoints 
\question Below is an attempt to write a procedure in C++ which doubles the value of the argument it is given:
\begin{displaymath}
\begin{array}{l}
\mathbf{void}~\mathit{dbl}(\mathbf{int}~x)~\{ \\
\quad x = x*2;\\
\}
\end{array}
\end{displaymath}
\begin{parts}
\part[3] Suppose that this procedure is called as follows:
\begin{displaymath}
\begin{array}{l}
\mathbf{int}~x = 23; \\
\mathit{dbl}(x); \\
\mathit{printf}(\texttt{"\%d"},x);
\end{array}
\end{displaymath}
What is the output of this program and why? The output of the program is likely not what the person who wrote $\mathit{dbl}$ intended. Change the definition of $\mathit{dbl}$ to produce the correct result, without changing the return type. What else do you need to change to make your revised procedure work? \droppoints 
\part[4] You now want to translate your revised $\mathit{dbl}$ procedure to Java. Show how this can be done. Explain your answer. \droppoints 
\end{parts}
\question[4] Suppose you have an array in Java which was declared using $\mathbf{int}[]~\mathit{test}$. We assume that all of the values in this array are initialised to 0. Write some Java code to demonstrate that $\mathit{test}$ is a reference and explain your answer. \droppoints 
\question Pointers may be set to $\mathit{NULL}$ to indicate that they are not pointing to anything useful.
\begin{parts}
\part[4] Discuss advantages and disadvantages of this approach when compared with the use of the option type in ML, whose definition is given below:
\begin{displaymath}
\mathbf{datatype}~'a~option = \mathit{NONE} \mid \mathit{SOME}~\mathbf{of}~'a
\end{displaymath} \droppoints 
\part[4] How do references relate to the above? \droppoints 
\end{parts}
\question[6] You have a friend who is a PhD student in the Cambridge Programming languages Research Group. He proposes a \emph{functional} language in which all variables are passed by reference. Discuss to what extent this makes sense (if at all) and why? \droppoints 

\section*{Classes}

\question Below is an ML representation of a 3D vector:
\begin{displaymath}
\mathbf{datatype}~\mathit{vector3D} = \mathit{V3D}~\mathbf{of}~\mathit{real} * \mathit{real} * \mathit{real};
\end{displaymath}
\begin{parts}
\part[4] Write a class in Java which corresponds to this data type. \droppoints 
\part[8] You are given the following function which adds up two vectors in ML:
\begin{displaymath}
\mathbf{fun}~\mathit{add}~(\mathit{V3D}(x,y,z))~(\mathit{V3D}(a,b,c)) = \mathit{V3D}(x+a,y+b,z+c);
\end{displaymath}
Now add an $\mathit{add}$ method to your vector class in Java. Explain at least \emph{four} different approaches to doing this. \droppoints 
\end{parts}
\question[4] Show how ML's $\mathit{option}$ type can be implemented in an object-oriented language using generics. \droppoints 
\question[10] Implement linked lists in Java. Your implementation should support adding/removing elements from the head of the list, retrieving the head (without removing it), and obtaining the $n^{\text{th}}$ element. \emph{Hint}: you may find it useful to define a $\mathit{LinkedList}$ class as well as a $\mathit{LinkedListItem}$ class. \droppoints 
\question[4] Discuss how linked lists in an object-oriented language like Java compare to linked lists in a functional language like ML. \droppoints 
\question[15] Implement binary trees in Java. Note that the binary trees do \emph{not} need to be balanced. Your implementation should support insertions and lookups. \droppoints 

\end{questions}
\end{document}